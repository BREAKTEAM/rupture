\section*{Appendix A: Full proofs}

\begin{lemma}[Compression attack]
\end{lemma}

\begin{proof}

Let $g$ be the boolean function $Q$ on the plaintext. Define the adversary
$\mathcal{A}$ as follows:

\begin{lstlisting}[texcl,mathescape,basicstyle=\small]
def $\mathcal{A}(Q, \mathcal{M}, y)$:
    $(r_1, r_2) \leftarrow \mathcal{O}_R(\textrm{Com}, f, Q, \mathcal{M})$

    $l_1 = |\text{Reflect}^{\mathcal{E}_{pk}}_s(r_1)|$
    $l_2 = |\text{Reflect}^{\mathcal{E}_{pk}}_s(r_2)|$

    if $l_1 < l_2$:
        return True
    else:
        return False
\end{lstlisting}

Let $\mathcal{S}$ be an arbitrary simulator. Then we have:
\begin{align*}
    \Pr[\text{Game}_{\text{REF-SIM}}^{\mathcal{PE},\mathcal{S}}
        (\lambda, f, \mathcal{M}, g) = 1] &=\\
    \Pr_{x \leftarrow \mathcal{M}, b \leftarrow \mathcal{S}(f, \mathcal{M}, g)}
        [Q(x) = b]
\end{align*}

Letting the random variables $b$ and $x$:
\begin{align*}
    x &\leftarrow \mathcal{M}\\
    b &\leftarrow \mathcal{S}(f, \mathcal{M}, g)
\end{align*}

Due to the independence of the simulator's output with the choice of $x$ we have:
\begin{align*}
    Pr[b = Q(x)] &=\\
    Pr[\lnot b|\lnot Q(x)]Pr[\lnot Q(x)] + Pr[b|Q(x)]Pr[Q(x)] &=\\
    Pr[\lnot b]Pr[\lnot Q(x)] + Pr[b]Pr[Q(x)] &=\\
    (1 - Pr[b])(1 - Pr[Q]) + Pr[b]Pr[Q(x)] &=\\
    1 - Pr[Q(x)] - Pr[b] + 2Pr[b]Pr[Q(x)]
\end{align*}

For a given $\Pr[Q]$, this function is monotonic in $\Pr[b]$ and therefore has
potential extrema at $\Pr[b] = 0$ or $\Pr[b] = 1$, for which cases the function
takes the values $1 - \Pr[Q(x)]$ and $\Pr[Q(x)]$ respectively. Therefore, the
maximum $\Pr[b]$ of the simulator is:
\begin{equation*}
    \Pr[Q(x) = b] = max(\Pr[Q(x)], 1 - \Pr[Q(x)]) = \pi
\end{equation*}

And therefore:
\begin{align*}
    \forall \mathcal{S}:\\
    \Pr[
        \text{Game}_{\text{REF-SIM}}^{\mathcal{PE},\mathcal{S}}
        (\lambda, f, \mathcal{M}, g) = 1
    ]
    \leq\\
    max(Pr[Q(x)], 1 - Pr[Q(x)])
\end{align*}

From the compression detectability of Q we know that:
\begin{align*}
    \exists \alpha \text{ non-negl}:\\
    \Pr_{s_1 \leftarrow \mathcal{M}_Q,
         s_2 \leftarrow \mathcal{M}_{\lnot Q}}
         [cpr^Q_{\kappa}(s_1, \overbar{r}) \land
          cpr^Q_{\kappa}(s_2, \overbar{r})]
    \geq\\
    \pi + \alpha(\lambda)
\end{align*}

Therefore:
\begin{align*}
    \Pr_{s_1 \leftarrow \mathcal{M}_Q}
         [cpr^Q_{\kappa}(s_1, \overbar{r})]
    \geq
    \pi + \alpha(\lambda) \land\\
    \Pr_{s_2 \leftarrow \mathcal{M}_{\lnot Q}}
         [cpr^Q_{\kappa}(s_2, \overbar{r})]
    \geq
    \pi + \alpha(\lambda)
\end{align*}

And so:
\begin{align*}
    \Pr_{s \leftarrow \mathcal{M}}
         [cpr^Q_{\kappa}(s, \overbar{r})]
    =\\
    \Pr[cpr^Q_{\kappa}(s, \overbar{r})|Q(s)]\Pr[Q(s)]
    +\\
    \Pr[cpr^Q_{\kappa}(s, \overbar{r})|\lnot Q(s)]\Pr[\lnot Q(s)]
    \geq\\
    (\pi + \alpha(\lambda))(\Pr[Q(s)] + (1 - \Pr[Q(s)))
    =\\
    \pi + \alpha(\lambda)
\end{align*}

Let us now examine the event of $\mathcal{A}$ being successful, denoted Succ,
when $cpr^Q_{\kappa}(s, \overbar{r})$. Assuming $Q(s)$:
\begin{align*}
    \Pr[|\mathcal{K}(s, r_1)| < |\mathcal{K}(s, r_2)||Q(s)]
    = \pi + \alpha(\lambda)
\end{align*}

And from the strict length monotonicity of $\textrm{Enc}$ it follows that:
\begin{align*}
    \Pr[&|\textrm{Enc}(\mathcal{K}(s, r_1))| <\\&|\textrm{Enc}(\mathcal{K}(s, r_2))||Q(s)]
        \geq \pi + \alpha(\lambda)\\
    \Rightarrow \Pr[&
        |\text{Reflect}^{\mathcal{E}_{pk}}_s(r_1)|
        <
        |\text{Reflect}^{\mathcal{E}_{pk}}_s(r_2)||Q(s)
    ]
        \geq \pi + \alpha(\lambda)\\
    \Rightarrow \Pr[&l_1 < l_2|Q(s)]
        \geq \pi + \alpha(\lambda)\\
    \Rightarrow \Pr[&\text{Succ}|Q(s)]
        \geq \pi + \alpha(\lambda)\\
\end{align*}

The case for $\lnot Q(s)$ is the same, but with a different inequality direction:
\begin{align*}
    \Pr[|\mathcal{K}(s, r_1)| > |\mathcal{K}(s, r_2)||\lnot Q(s)]
        \geq\\
        \pi + \alpha(\lambda)\\
    \Rightarrow \Pr[\text{Succ}|\lnot Q(s)] \geq\\
    \pi + \alpha(\lambda)\\
\end{align*}

And so the probability of success is given:
\begin{align*}
    \Pr[\text{Succ}] =\\
    \Pr[\text{Succ}|Q(s)]\Pr[Q(s)]
    +
    \Pr[\text{Succ}|\lnot Q(s)]\Pr[\lnot Q(s)] \geq\\
    \pi + \alpha(\lambda)
\end{align*}

Therefore,
\begin{align*}
    \forall PPT \mathcal{S}:
    \text{Adv}_{\mathcal{SE}(\textrm{Enc}, \textrm{Com}), \mathcal{A}, \mathcal{S}}
        (\lambda, f, \mathcal{M}, g)
    \geq\\
    |\pi + \alpha(\lambda) - \pi| = \alpha(\lambda)
\end{align*}

Which is non-negligible.

\end{proof}

\begin{lemma}[Amplification]
\end{lemma}

\begin{proof}

From the fact that $Q$ is compression-detectable and from the Compression Attack Theorem, we have that:
\begin{align*}
    \Pr_{s \leftarrow \mathcal{M}}
         [cpr^Q_{\kappa}(s, \overbar{r})]
    =\\
    \pi + \alpha(\lambda)
\end{align*}

Some elements $s \in \mathcal{M}$ allow for better compression-detectability than others under the fixed
reflection vector $\overbar{r}$. Call these elements \textit{amplifiable} and define predicate:
\begin{align*}
    Amp(s) \defeq
    \Pr[cpr^Q_{\kappa}
     (s, \overbar{r})]
    \geq
    \frac{1}{2} + \frac{\alpha(\lambda)}{2}
\end{align*}

We will now obtain a lower bound on the probability of an element being amplifiable.

Let:

% B derivation: (where \beta(\lambda) = \alpha(\lambda) / 2)
% B + (\frac{1}{2} + \alpha(\lambda)/2)(1 - B) = \pi + \alpha(\lambda)
% B + \frac{1}{2} + \alpha(\lambda)/2 - B(\frac{1}{2} + \beta(\lambda)) = \pi + \alpha(\lambda)
% \frac{1}{2} + \beta(\lambda) + B(\frac{1}{2} - \beta(\lambda)) = \pi + \alpha(\lambda)
% B(\frac{1}{2} - \beta(\lambda)) = \pi + \alpha(\lambda) - \frac{1}{2} - \beta(\lambda)
% B = (\pi + \alpha(\lambda) - \frac{1}{2} - \beta(\lambda)) / (\frac{1}{2} - \beta(\lambda))
% B = (\pi + \alpha(\lambda) - \frac{1}{2} - \frac{\alpha(\lambda)}{2}) / (\frac{1}{2} - \frac{\alpha(\lambda)}{2})
% B = (\pi - \frac{1}{2} + \frac{\alpha(\lambda)}{2}) / (\frac{1}{2} - \frac{\alpha(\lambda)}{2})
% B = \pi / (\frac{1}{2} - \frac{\alpha(\lambda)}{2}) - (\frac{1}{2} - \frac{\alpha(\lambda)}{2}) / (\frac{1}{2} - \frac{\alpha(\lambda)}{2})
% B = \pi / (\frac{1}{2} - \frac{\alpha(\lambda)}{2}) - 1
\begin{align*}
    B = \frac{\pi}{\frac{1}{2} - \frac{\alpha(\lambda)}{2}} - 1
\end{align*}

$B$ is non-negligible in $\lambda$.

Assume, for the sake of contradiction, that:
\begin{align*}
    \Pr_{s \leftarrow \mathcal{M}}
    [Amp(s)] < B
\end{align*}

Then we have:
\begin{align*}
    \Pr_{s \leftarrow \mathcal{M}}
         [cpr^Q_{\kappa}(s, \overbar{r})]
    =\\
    \Pr_{s \leftarrow \mathcal{M}}
         [cpr^Q_{\kappa}(s, \overbar{r})|Amp(s)]\Pr[Amp(s)]
    +\\
    \Pr_{s \leftarrow \mathcal{M}}
         [cpr^Q_{\kappa}(s, \overbar{r})|\lnot Amp(s)]\Pr[\lnot Amp(s)]
    <\\
    B + (\frac{1}{2} + \frac{\alpha}{2})(1 - B)
\end{align*}

But then:
\begin{align*}
    B + (\frac{1}{2} + \frac{\alpha(\lambda)}{2})(1 - B) =
    \pi + \alpha(\lambda)
\end{align*}

And this contradicts the assumption that $Q$ is compression-detectable. Therefore:
\begin{align*}
    \Pr_{s \leftarrow \mathcal{M}}
    [Amp(s)] \geq B
\end{align*}

It remains to show that the advantage of the adversary can be
arbitrarily large, i.e. that for some negligible $C$ we have:
\begin{align*}
    \text{Adv}_{\mathcal{SE}(\textrm{Enc}, \textrm{Com}), \mathcal{A}, \mathcal{S}_{Amp}}
    (\lambda, f, \mathcal{M}, g) = 1 - \pi - C
\end{align*}

Indeed, observe that the amplifying adversary performs a repeated Bernoulli
trial with $k$ repetitions and extracts a majority. Let $X$ be the number of
repetitions that are successful for the adversary. $X$ is defined:
\begin{align*}
    X \defeq |\{ i: cpr^Q_{\kappa}(s, \overbar{r}) \}|
\end{align*}

$X$ follows the binomial distribution, and therefore its expected value is:
\begin{align*}
    E[X] = \frac{k}{2} + \frac{\alpha k}{2}
\end{align*}

Let Succ denote the event of the amplified adversary succeeding. Then Succ
is equivalent to $X > \frac{k}{2}$.

Because $\alpha$ is non-negligible and due to the tail bounds of the binomial
distribution, we have:
\begin{align*}
    Pr[Succ] = 1 - C(k)
\end{align*}

Where $C$ is a negligible function.
\end{proof}
