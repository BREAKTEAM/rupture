\section*{Appendix A: Full proofs}

\begin{lemma}[Semantic security]
\end{lemma}

\begin{proof}
    We will prove that if the scheme is not semantically secure, then it is
    necessarily not reflection secure. Assume that the scheme is not
    semantically secure. Then there will exist a PPT adversary $\mathcal{A}$
    such that for all PPT simulators $\mathcal{S}$ we have that $\mathcal{A}$
    has non-negligible advantage to $\mathcal{S}$ in the semantic security game.

    We will now construct an adversary $\mathcal{A'}$ for the reflection
    security game. $\mathcal{A'}$ operates as follows: It makes one query to
    the reflection oracle setting $r_0 = \epsilon$ and receives a response $c_0
    = K_k(s, r_0) = K_k(s) = c$. It then passes that $c$ to the semantic
    security adversary $\mathcal{A}$. It answers all encryption and decryption
    queries of the semantic security adversary by relaying them to its own
    encryption and decryption oracle. Finally, when the semantic security
    adversary outputs $y = y'$, this $y'$ is output by the reflection security
    adversary. Note that $\mathcal{A}$ cannot distinguish whether they are
    playing against the actual semantic security game or being simulated by the
    reflection security adversary, as their view is identical. Therefore:

    \begin{equation}
        Pr[Game_{REF-SEC}^{\mathcal{A'}} = 1] = Pr[Game_{SEM-SEC}^{\mathcal{A}} = 1]
    \end{equation}

    It remains to prove that $\mathcal{A'}$ has significant advantage against
    any reflection simulator $\mathcal{S'}$. Indeed let $\mathcal{S'}$ be any
    reflection game simulator. We will construct a simulator $\mathcal{S}$ for
    the semantic security game. Initially, the simulator $\mathcal{S}$ receives
    the length of the ciphertext $|c|$. They then simulate the reflection game
    simulator $\mathcal{S'}$ as follows. Upon receiving query $r_i$ from the
    reflection game simulator, they answer with $|c| + |r_i|$. When the
    reflection security adversary outputs $y' = y$, this $y$ is output by the
    semantic security simulator. We observe that the reflection game simulator
    $\mathcal{S'}$ cannot distinguish whether they are playing against the
    actual simulated reflection game or being simulated by the semantic
    security simulator. To see this, note that from the length preservation
    assumption we have that $|c| + |r_i| = |K(s, r_i)| = |K(s', r_i)| =
    |K(0^{|s|}, r_i)|$. Therefore:

    \begin{equation}
        Pr[Game_{SEM-SIM}^{\mathcal{S}} = 1] = Pr[Game_{REF-SIM}^{\mathcal{S'}} = 1]
    \end{equation}

    From the assumption that the scheme is not semantically insecure, we know
    that:

    \begin{align*}
        |Pr[Game_{SEM-SEC}^{\mathcal{A}} = 1] - Pr[Game_{SEM-SIM}^{\mathcal{S}} = 1]| =\\
        Adv_{\mathcal{A}, \mathcal{S}} = \text{non-negl}
    \end{align*}

    And therefore, replacing both probabilities with their equals:

    \begin{align*}
        |Pr[Game_{REF-SEC}^{\mathcal{A'}} = 1] -
        Pr[Game_{REF-SIM}^{\mathcal{S'}} = 1]| =\\
        Adv_{\mathcal{A'}, \mathcal{S'}} = \text{non-negl}
    \end{align*}
\end{proof}

\begin{lemma}[Good compression is detectable]
\end{lemma}

\begin{proof}
We model our plaintext input to the compression function as the usual pair $(s,
r)$ consisting of the secret string $s$ and reflection string $r$. When $(s, r)$
are encoded by a compression function $\mathcal{K}$ ideal under some plaintext
distribution $\bar{\mathcal{M}}$, a simple predicate allows the distinction
between two secrets $s_1$ and $s_2$ using a reflection pair $(r_1, r_2)$.  We
make the assumption that the resulting reflection pair $(r_1, r_2)$ is
efficiently computable.

\begin{align*}
    \Pr[r = r_1|s = s_1] < \Pr[r = r_2|s = s_1]&\land\\
    \Pr[r = r_1|s = s_2] > \Pr[r = r_2|s = s_2]&
\end{align*}

Using the fact that $\mathcal{K}$ is ideal with respect to this distribution,
we deduce that:

\begin{align*}
    |\mathcal{K}(s_1, r_1)| > |\mathcal{K}(s_1, r_2)|&\land\\
    |\mathcal{K}(s_2, r_1)| < |\mathcal{K}(s_2, r_2)|&
\end{align*}

We then define $Q(s)$ to be the predicate ``$s = s_1$". This partitions
$\mathcal{M}$ into the distributions $\mathcal{M}_Q$ and
$\mathcal{M}_{\lnot Q}$ both of which are non-empty, as $\mathcal{M}_Q$
contains $s_1$ and $\mathcal{M}_{\lnot Q}$ contains $s_2$. From the fact that
$Q$ is not trivial, we deduce that $\pi < 1$.

Observe, then, that letting $\bar{r} = (r_1, r_2)$ we obtain:

\begin{align*}
    \Pr[cpr^Q_{\mathcal{K}}(s_1, \bar{r}, \mathcal{K}, \mathcal{M})
    \land
    cpr^Q_{\mathcal{K}}(s_2, \bar{r}, \mathcal{K}, \mathcal{M})] = 1
\end{align*}

This completes the proof.

\end{proof}

\begin{lemma}[Compression attack]
\end{lemma}

\begin{proof}

Let $g$ be the boolean function $Q$ on the plaintext. Define the adversary
$\mathcal{A}$ as follows:

\begin{lstlisting}[texcl,mathescape,basicstyle=\small]
def $\mathcal{A}(1^\lambda)$:
    $(r_1, r_2) \leftarrow \mathcal{O}_R(1^\lambda)$

    $l_1 = |\text{Reflect}^{k}_s(r_1)|$
    $l_2 = |\text{Reflect}^{k}_s(r_2)|$

    if $l_1 < l_2$:
        return True
    else:
        return False
\end{lstlisting}

Let $\mathcal{S}$ be an arbitrary simulator. Then we have:
\begin{align*}
    \Pr[\text{Game}_{\text{REF-SIM}}^{\mathcal{SE},\mathcal{S}}
        (\lambda) = 1] &=\\
    \Pr_{x \leftarrow \mathcal{M}, b \leftarrow \mathcal{S}(1^\lambda)}
        [Q(x) = b]
\end{align*}

Letting the random variables $b$ and $x$:
\begin{align*}
    x &\leftarrow \mathcal{M}\\
    b &\leftarrow \mathcal{S}(1^\lambda)
\end{align*}

Due to the independence of the simulator's output with the choice of $x$ we have:
\begin{align*}
    Pr[b = Q(x)] &=\\
    Pr[\lnot b|\lnot Q(x)]Pr[\lnot Q(x)] + Pr[b|Q(x)]Pr[Q(x)] &=\\
    Pr[\lnot b]Pr[\lnot Q(x)] + Pr[b]Pr[Q(x)] &=\\
    (1 - Pr[b])(1 - Pr[Q]) + Pr[b]Pr[Q(x)] &=\\
    1 - Pr[Q(x)] - Pr[b] + 2Pr[b]Pr[Q(x)]
\end{align*}

For a given $\Pr[Q]$, this function is monotonic in $\Pr[b]$ and therefore has
potential extrema at $\Pr[b] = 0$ or $\Pr[b] = 1$, for which cases the function
takes the values $1 - \Pr[Q(x)]$ and $\Pr[Q(x)]$ respectively. Therefore, the
maximum $\Pr[b]$ of the simulator is:
\begin{equation*}
    \Pr[Q(x) = b] = max(\Pr[Q(x)], 1 - \Pr[Q(x)]) = \pi
\end{equation*}

And therefore:
\begin{align*}
    \forall \mathcal{S}:\\
    \Pr[
        \text{Game}_{\text{REF-SIM}}^{\mathcal{SE},\mathcal{S}}
        (\lambda) = 1
    ]
    \leq\\
    max(Pr[Q(x)], 1 - Pr[Q(x)])
\end{align*}

From the compression detectability of Q we know that:
\begin{align*}
    \exists \alpha \text{ non-negl}:\\
    \Pr_{s_1 \leftarrow \mathcal{M}_Q,
         s_2 \leftarrow \mathcal{M}_{\lnot Q}}
         [cpr^Q_{\kappa}(s_1, \overbar{r}) \land
          cpr^Q_{\kappa}(s_2, \overbar{r})]
    \geq\\
    \pi + \alpha(\lambda)
\end{align*}

Therefore:
\begin{align*}
    \Pr_{s_1 \leftarrow \mathcal{M}_Q}
         [cpr^Q_{\kappa}(s_1, \overbar{r})]
    \geq
    \pi + \alpha(\lambda) \land\\
    \Pr_{s_2 \leftarrow \mathcal{M}_{\lnot Q}}
         [cpr^Q_{\kappa}(s_2, \overbar{r})]
    \geq
    \pi + \alpha(\lambda)
\end{align*}

And so:
\begin{align*}
    \Pr_{s \leftarrow \mathcal{M}}
         [cpr^Q_{\kappa}(s, \overbar{r})]
    =\\
    \Pr[cpr^Q_{\kappa}(s, \overbar{r})|Q(s)]\Pr[Q(s)]
    +\\
    \Pr[cpr^Q_{\kappa}(s, \overbar{r})|\lnot Q(s)]\Pr[\lnot Q(s)]
    \geq\\
    (\pi + \alpha(\lambda))(\Pr[Q(s)] + (1 - \Pr[Q(s)))
    =\\
    \pi + \alpha(\lambda)
\end{align*}

Let us now examine the event of $\mathcal{A}$ being successful, denoted Succ,
when $cpr^Q_{\kappa}(s, \overbar{r})$. Assuming $Q(s)$:
\begin{align*}
    \Pr[|\mathcal{K}(s, r_1)| < |\mathcal{K}(s, r_2)||Q(s)]
    = \pi + \alpha(\lambda)
\end{align*}

And from the strict length monotonicity of $\textrm{Enc}$ it follows that:
\begin{align*}
    \Pr[&|\textrm{Enc}(\mathcal{K}(s, r_1))| <\\&|\textrm{Enc}(\mathcal{K}(s, r_2))||Q(s)]
        \geq \pi + \alpha(\lambda)\\
    \Rightarrow \Pr[&
        |\text{Reflect}^{k}_s(r_1)|
        <
        |\text{Reflect}^{k}_s(r_2)||Q(s)
    ]
        \geq \pi + \alpha(\lambda)\\
    \Rightarrow \Pr[&l_1 < l_2|Q(s)]
        \geq \pi + \alpha(\lambda)\\
    \Rightarrow \Pr[&\text{Succ}|Q(s)]
        \geq \pi + \alpha(\lambda)\\
\end{align*}

The case for $\lnot Q(s)$ is the same, but with a different inequality direction:
\begin{align*}
    \Pr[|\mathcal{K}(s, r_1)| > |\mathcal{K}(s, r_2)||\lnot Q(s)]
        \geq\\
        \pi + \alpha(\lambda)\\
    \Rightarrow \Pr[\text{Succ}|\lnot Q(s)] \geq\\
    \pi + \alpha(\lambda)\\
\end{align*}

And so the probability of success is given:
\begin{align*}
    \Pr[\text{Succ}] =\\
    \Pr[\text{Succ}|Q(s)]\Pr[Q(s)]
    +
    \Pr[\text{Succ}|\lnot Q(s)]\Pr[\lnot Q(s)] \geq\\
    \pi + \alpha(\lambda)
\end{align*}

Therefore,
\begin{align*}
    \forall PPT \mathcal{S}:
    \text{Adv}_{\mathcal{SE}(\textrm{Enc}, \textrm{Com}), \mathcal{A}, \mathcal{S}}
        (1^\lambda)
    \geq\\
    |\pi + \alpha(\lambda) - \pi| = \alpha(\lambda)
\end{align*}

Which is non-negligible.

\end{proof}

\begin{lemma}[Amplification]
\end{lemma}

\begin{proof}

From the fact that $Q$ is compression-detectable and from the Compression Attack Theorem, we have that:
\begin{align*}
    \Pr_{s \leftarrow \mathcal{M}}
         [cpr^Q_{\kappa}(s, \overbar{r})]
    =\\
    \pi + \alpha(\lambda)
\end{align*}

Some elements $s \in \mathcal{M}$ allow for better compression-detectability than others under the fixed
reflection vector $\overbar{r}$. Call these elements \textit{amplifiable} and define predicate:
\begin{align*}
    Amp(s) \defeq
    \Pr[cpr^Q_{\kappa}
     (s, \overbar{r})]
    \geq
    \frac{1}{2} + \frac{\alpha(\lambda)}{2}
\end{align*}

We will now obtain a lower bound on the probability of an element being amplifiable.

Let:

% B derivation: (where \beta(\lambda) = \alpha(\lambda) / 2)
% B + (\frac{1}{2} + \alpha(\lambda)/2)(1 - B) = \pi + \alpha(\lambda)
% B + \frac{1}{2} + \alpha(\lambda)/2 - B(\frac{1}{2} + \beta(\lambda)) = \pi + \alpha(\lambda)
% \frac{1}{2} + \beta(\lambda) + B(\frac{1}{2} - \beta(\lambda)) = \pi + \alpha(\lambda)
% B(\frac{1}{2} - \beta(\lambda)) = \pi + \alpha(\lambda) - \frac{1}{2} - \beta(\lambda)
% B = (\pi + \alpha(\lambda) - \frac{1}{2} - \beta(\lambda)) / (\frac{1}{2} - \beta(\lambda))
% B = (\pi + \alpha(\lambda) - \frac{1}{2} - \frac{\alpha(\lambda)}{2}) / (\frac{1}{2} - \frac{\alpha(\lambda)}{2})
% B = (\pi - \frac{1}{2} + \frac{\alpha(\lambda)}{2}) / (\frac{1}{2} - \frac{\alpha(\lambda)}{2})
% B = \pi / (\frac{1}{2} - \frac{\alpha(\lambda)}{2}) - (\frac{1}{2} - \frac{\alpha(\lambda)}{2}) / (\frac{1}{2} - \frac{\alpha(\lambda)}{2})
% B = \pi / (\frac{1}{2} - \frac{\alpha(\lambda)}{2}) - 1
\begin{align*}
    B = \frac{\pi}{\frac{1}{2} - \frac{\alpha(\lambda)}{2}} - 1
\end{align*}

$B$ is non-negligible in $\lambda$.

Assume, for the sake of contradiction, that:
\begin{align*}
    \Pr_{s \leftarrow \mathcal{M}}
    [Amp(s)] < B
\end{align*}

Then we have:
\begin{align*}
    \Pr_{s \leftarrow \mathcal{M}}
         [cpr^Q_{\kappa}(s, \overbar{r})]
    =\\
    \Pr_{s \leftarrow \mathcal{M}}
         [cpr^Q_{\kappa}(s, \overbar{r})|Amp(s)]\Pr[Amp(s)]
    +\\
    \Pr_{s \leftarrow \mathcal{M}}
         [cpr^Q_{\kappa}(s, \overbar{r})|\lnot Amp(s)]\Pr[\lnot Amp(s)]
    <\\
    B + (\frac{1}{2} + \frac{\alpha}{2})(1 - B)
\end{align*}

But then:
\begin{align*}
    B + (\frac{1}{2} + \frac{\alpha(\lambda)}{2})(1 - B) =
    \pi + \alpha(\lambda)
\end{align*}

And this contradicts the assumption that $Q$ is compression-detectable. Therefore:
\begin{align*}
    \Pr_{s \leftarrow \mathcal{M}}
    [Amp(s)] \geq B
\end{align*}

It remains to show that the advantage of the adversary can be
arbitrarily large, i.e. that for some negligible $C$ we have:
\begin{align*}
    \text{Adv}_{\mathcal{SE}(\textrm{Enc}, \textrm{Com}), \mathcal{A}, \mathcal{S}_{Amp}}
    (1^\lambda) = 1 - \pi - C
\end{align*}

Indeed, observe that the amplifying adversary performs a repeated Bernoulli
trial with $k$ repetitions and extracts a majority. Let $X$ be the number of
repetitions that are successful for the adversary. $X$ is defined:
\begin{align*}
    X \defeq |\{ i: cpr^Q_{\kappa}(s, \overbar{r}) \}|
\end{align*}

$X$ follows the binomial distribution, and therefore its expected value is:
\begin{align*}
    E[X] = \frac{k}{2} + \frac{\alpha k}{2}
\end{align*}

Let Succ denote the event of the amplified adversary succeeding. Then Succ
is equivalent to $X > \frac{k}{2}$.

Because $\alpha$ is non-negligible and due to the tail bounds of the binomial
distribution, we have:
\begin{align*}
    Pr[Succ] = 1 - C(k)
\end{align*}

Where $C$ is a negligible function.
\end{proof}

\begin{lemma}[Reflection-independent security]
    Let $E$ be a length-preserving semantically secure encryption schema and
    $Com$ be any function. Then $E(s)$ is reflection secure.
\end{lemma}

\begin{proof}
    Let $\mathcal{A}$ be a reflection security adversary for $E(s)$. We will
    construct a semantic security adversary $\mathcal{A'}$ for $E$. Our semantic
    security adversary will break semantic security for \textit{multiple
    messages}. Let $\mathcal{M}$ be the distribution that $\mathcal{A}$
    attacks. Because $\mathcal{A}$ is a probabilistic polynomial-time
    adversary, there is a polynomial bound $t(\lambda)$ for its execution time.
    The number of reflection queries conducted by $\mathcal{A}$ is then also
    bound by $t(\lambda)$.

    Define $\mathcal{M'}$ by the following procedure: First, $s$ is chosen
    out of the distribution $\mathcal{M}$. Then that same $s$ is repeated
    $t(\lambda)$ times to obtain the vector of plaintexts $\overbar{s}$.

    Upon execution, $\mathcal{A'}$ receives from its challenger the encrypted
    vector $\overbar{c}$ where the same plaintext has been independently
    encrypted $t(\lambda)$ times.
    $\mathcal{A'}$ answers encryption and decryption queries of $\mathcal{A}$
    by forwarding them to its own encryption and
    decryption oracles. For reflection queries, because the schema trivially
    ignores $r$ since $E(s)$ is independent of $r$, the semantic security
    adversary is able to answer each query $r_i$ with $c_i$.

    Clearly, the view of the simulated reflection adversary is the same as if
    it were run in an actual reflection game. Therefore we have:

    \begin{align*}
        Pr[\text{Game}_{\text{REF-SEC}}^{\mathcal{A}}(1^{\lambda})
        = 1]
        =
        Pr[\text{Game}_{\text{SEM-SEC}}^{\mathcal{A'}}(1^{\lambda})
        = 1]
    \end{align*}

    We now prove that the adversary $\mathcal{A'}$ obtains a non-negligible gap
    against any semantic security simulator $\mathcal{S'}$. Indeed let
    $\mathcal{S'}$ be any semantic security simulator. We will construct a
    reflection security simulator $\mathcal{S}$. $\mathcal{S}$ works as
    follows: It obtains $|Com(s)|$ from its challenger which it ignores. It
    then issues a single reflection query $r_1 = \epsilon$, to which the
    challenger responds with $c = E(s')$. But $|c| = |E(s')| = |s'| =
    |0^{|s|}| = |s| = |E(s)| = |c|$ due to length preservation. This $|c|$ is
    then used to start the semantic security simulator.

    The view of the simulated semantic simulator is the same as if it were run
    in an actual semantic simulation game. Therefore we have:

    \begin{align*}
        Pr[\text{Game}_{\text{REF-SIM}}^{\mathcal{S}}(1^{\lambda})
        = 1]
        =
        Pr[\text{Game}_{\text{SEM-SIM}}^{\mathcal{S'}}(1^{\lambda})
        = 1]
    \end{align*}

    Because there exists a non-negligible advantage between $\mathcal{A}$ and
    any simulator $\mathcal{S}$, this gap will also exist between the
    constructed simulator $\mathcal{S}$. Therefore, we have:

    \begin{align*}
        \text{non-negl} = \text{Adv}_{\text{REF} \mathcal{A}, \mathcal{S}}\\
        =
        Pr[\text{Game}_{\text{REF-SEC}}^{\mathcal{A}}
        = 1] -
        Pr[\text{Game}_{\text{REF-SIM}}^{\mathcal{S}}
        = 1]\\
        =
        Pr[\text{Game}_{\text{SEM-SEC}}^{\mathcal{A'}}
        = 1]
        -
        Pr[\text{Game}_{\text{SEM-SIM}}^{\mathcal{S'}}
        = 1]\\
        =
        \text{Adv}_{\text{SEM} \mathcal{A'}, \mathcal{S'}}
    \end{align*}

    For clarity, we have omitted the security parameters from the games.
    This completes the proof.
\end{proof}

\begin{lemma}
    Under the above assumptions, $E(s) || r$ is reflection secure.
\end{lemma}

\begin{proof}
    The construction is identical to the previous proof. However, the
    reflection queries of the reflection adversary $\mathcal{A}$ must be
    answered correctly. As $r_i$ is known to the semantic security adversary,
    this can be done in the obvious way by setting $c_i' = c_i || r_i$.

    The reflection security simulator is not affected, as it receives $|c| =
    |Enc(s) || \epsilon| = |Enc(s)|$.
\end{proof}

\begin{lemma}
    Under the above assumptions and letting $Com$ be a deterministic
    efficiently computable function, $E(s) || Com(r)$ is reflection secure.
\end{lemma}

\begin{proof}
    The construction is the same as before. Because the semantic security
    adversary has access to the $Com$ function, it can calculcate $Com(r_i)$
    upon receiving the query $r_i$. Then setting $c_i = E(s) || Com(r_i)$ it
    responds as usual.

    The reflection simulator on the other hand will be affected by this
    construction. Indeed $|c| = |E(s') || Com(\epsilon)|$. However,
    $|Com(\epsilon)|$ can be computed by the simulator, and therefore the value
    given to the semantic simulator can be obtained as $|c'| = |E(s)| = |E(s')|
    = |c| - |Com(\epsilon)|$.
\end{proof}

\begin{lemma}
    Under the above assumptions and additionally requiring that $Com$ is a
    bijection which is also efficiently reversible, $E(Com(s))$ is reflection secure.
\end{lemma}

\begin{proof}
    This construction is slightly more complicated. Assuming $\mathcal{A}$ is a
    reflection-security adversary which works against the distribution
    $\mathcal{M}$ we will construct a multiple messages semantic security
    adversary as before. However, the distribution $\mathcal{M'}$ will now be
    obtained by the following procedure: Initially, $s$ is chosen out of
    $\mathcal{M}$. Then $Com(s)$ is calculated. This value is then repeated
    $t(\lambda)$ times. Our semantic security adversary will work against
    $\mathcal{M'}$.

    When the challenger produces the encrypted vector $\overbar{c}$, it will
    contain $t(\lambda)$ encryptions of the plaintext $Com(s)$. These can
    readily now be passed to the reflection adversary as reflection query
    responses as before.

    The reflection security simulator now makes use of the $|Com(s)|$ input,
    which it uses to start up the semantic security simulator.

    We now argue that the advantage obtained pertains to a valid predicate on
    the compressed plaintexts space. Indeed, without loss of generality assume
    that $g$ is a predicate. Then $g$ partitions the plaintext space into two
    partitions. Due to $Com$ being a bijection, these partitions are preserved
    after compression and the plaintext predicate $g$ is isomorphic to a
    compressed text predicate $g'$. Therefore, the semantic security adversary
    is able to obtain a non-negligible advantage for a different predicate.
\end{proof}

\begin{lemma}
    Under the above assumptions, $E(Com(s)) || r)$, $E(Com(s)) || E(r))$, and
    $E(Com(s)) || E(Com(r))$ are all reflection secure.
\end{lemma}

\begin{proof}
    The proofs for the above schemas can be combined in the obvious way to
    obtain the result needed.
\end{proof}

\begin{lemma}
    Under the above assumptions, $E(s || r)$ is reflection secure.
\end{lemma}

\begin{proof}
    We provide a sketch of this proof.

    To obtain this proof, we construct a semantic security adversary which
    works against a distribution $\mathcal{M'}$ defined as follows. First, $s$
    is chosen out of $\mathcal{M}$. Because the reflection security adversary
    is bounded in time by $t(\lambda)$, therefore each of the reflection
    queries $r_i$ must be $|r_i| < t(\lambda)$. The distribution $\mathcal{M'}$
    then will contain $t(\lambda)$ different vectors, each of dimension
    $t(\lambda)$. We require that the $j$-th $t$-dimensional vector contains
    the same secret text $s$ and an independently chosen reflection text $r$
    with $|r| = j$.

    Then the semantic security adversary $\mathcal{A'}$ works as follows: Upon
    receiving the reflection query $r_i$ from the reflection adversary, it
    responds with a fresh element from the $i$-th vector, which contains the
    encryption of an incorrect reflection string chosen uniformly at random. We
    call these \textit{mock} reflections, and notice that the mock reflection
    has the same length as the actual reflection query.

    We now argue that either this semantic security adversary is able to obtain
    a valid predicate for $s$, or that we have obtained a distinguisher for the
    value $r$.

    To do this, invoke a hybrid argument on the sequence of adversaries $B_i$
    such that the $i$-th adversary works as follows: It responds to the
    reflection queries $r_1, r_2, \cdots, r_i$ as $\mathcal{A'}$ would,
    using the mock reflections, but responds to the queries $r_{i+1},
    \cdots, r_{t(\lambda)}$ using the correct reflections. Clearly,
    $\mathcal{A'} = B_{t(\lambda)}$. If for all $j$ we have that $B_j$'s output
    is computationally indistinguishable from $B_{j+1}$'s output, we are done,
    since this would allow the construction of $\mathcal{A'}$ that obtains a
    non-negligible advantage. However, if for some $j$ we have that $B_j$'s
    output is computationally distinguishable from $B_{j+1}$, then this
    directly allows the construction of a semantic security adversary which is
    able to distinguish between different $r$ values. The intuition behind this
    is that since the mock and actual reflection queries have the same length,
    the assumption of semantic security should not allow the reflection
    adversary to tell apart the two reflection responses.
\end{proof}

\begin{lemma}
    Under the above assumptions, $E(Com(s) || Com(r))$ is reflection secure.
\end{lemma}

\begin{proof}
    The above lemma follows directly from the fact that $E(s || r)$ is
    reflection secure using the techniques previously used to obtain the
    reflection security of $E(Com(s))$ and $E(Com(s)) || E(Com(r))$. For the
    compression of the secret, the key intuition is that the reflection
    adversary has access to $|Com(s)|$, that the predicate attacked will be an
    isomorphic predicate on compression space instead of plaintext space, and
    that the distribution attacked will be the compression distribution instead
    of the plaintext distribution. This is readily obtained from the fact that
    $Com$ is a bijection. For the $Com(r)$ portion, the only
    requirement is that the compression function is accessible to both the
    semantic security adversary as well as the reflection security simulator,
    which is true under the assumption that $Com(r)$ is deterministic.
\end{proof}
