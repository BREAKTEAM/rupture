\begin{abstract}
When compression is composed with encryption, unexpected vulnerabilities can
arise. Such vulnerabilities have been explored in practical attacks against TLS
such as CRIME, TIME, and BREACH, even after TLS 1.3. Motivated by the lack of
production-grade testing tools, we provide Rupture, a modular scalable
open-source generic attack framework that implements compression side-channel
attacks in a robust manner. We introduce an abstract theoretical cryptographic
model to express such attacks, the \textit{adaptive reflection security} game
with the relevant security definitions. We propose a model for
\textit{compression idealness} and experimentally show that typical compression
functions expose partially ideal behavior. We then experimentally show that
typical plaintexts used in practice follow an interdependent joint distribution,
a stronger notion of dependent joint distributions. Based on these two
properties, we prove that \textit{compression-detectability} of predicates
arises, i.e. the ability to extract a predicate of the plaintext using
reflection. We then prove that all length-preserving encryption schemes are
insecure when composed with such functions. Our attack model describes two modes
of attack, one-shot and amplified. Finally, we propose a novel defense protocol,
\textit{context hiding}, that effectively eliminates the threat these attacks
pose. This protocol modifies the plaintext joint distribution in order to remove
the interdependence and mitigate the problem. We show that our method of defense
performs significantly more favourably in terms of compression rate compared to
previous techniques.
\end{abstract}
