\section*{Appendix B: Related work}\label{sec:related}

\subsection{Compression attacks in literature}

The compression side-channel attack technique was first studied by Kelsey
\cite{kelsey2002compression} in 2002.
Kelley and Tamassia\cite{kelley2014secure} discuss the notion of \textit{entropy-restricted} semantic
security. However, this only models the case where two secrets compress to the same
length, making it difficult to capture practical techniques used in attacks
such as BREACH.

Alawatugoda et al. \cite{alawatugoda2015protecting} introduce a model for \textit{Cookie recovery},
\textit{Random cookie indistinguishability}, and \textit{Chosen cookie indistinguishability}.
Nevertheless, their security models are limited in classes of rendering functions
that are very specific. In their models, the attacker is able to directly influence
the output of the rendering function through concatenation. More information on
rendering functions is provided in section \ref{subsec:terms}.

\subsection{Defenses}

Literature includes many proposed defenses for side-channel compression attacks
and especially BREACH, some of which are explored in the following sections.

\subsubsection{Disabling compression}\label{subsec:disablecom}
Compression is the basic assumption for the compression detectability. Disabling
compression would break the adaptive reflection game.

However, this is not a viable solution. Disabling compression results in a
performance penalty that outweighs the benefits of the defense.

\subsubsection{SameSite cookies}\label{subsec:samesite}
SameSite cookies \cite{mwest2015firstparty} is a proposed draft for enhancing
web security. The main puprose of this technique is to mitigate cross-site
request forgery attacks, although it also mitigates other kinds of cross-origin
leakages.

This policy introduces a new attribute for web cookies that web servers may
opt-in. This attribute states that cookies may only be included in a
``first-party" context, i.e. in same origin requests.

However, the current BREACH application depends on cross-origin requests in
order to call the reflection oracle. Enabling SameSite cookies would result in
the elimination of this reflection oracle of the Reflection-Security Game.

\subsubsection{Masking secrets}\label{subsec:masking}
Masking is a method of hiding secrets' properties.

A mask is a bitstream equal in size with the secret. The masked text contains
the result of the XOR operation between the secret and the mask. The masked
secret is then concatenated with the mask that was used. The secret can be
obtained by applying XOR between the masked secret and the mask.

A new mask is used for each request. That way, the secret's content is secured
against instances of the attack that require at least two calls to the
reflection oracle.

However, masking doubles the size of the protected secret. Furthermore, the mask
is random, and so is the masked secret, so the compression performance is poor.
Therefore, masking should be applied on high value secrets only, in order to
avoid significant performance penalties.

Masking is a method used by Facebook \cite{facebookbreach} in order to protect CSRF tokens
against BREACH attacks.
